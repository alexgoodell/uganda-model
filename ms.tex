%%%%%%%%%%%%%%%%%%%%%%%%%%%%%%%%%%%%%%%%%
% Simple Sectioned Essay Template
% LaTeX Template
%
% This template has been downloaded from:
% http://www.latextemplates.com
%
% Note:
% The \lipsum[#] commands throughout this template generate dummy text
% to fill the template out. These commands should all be removed when 
% writing essay content.
%
%%%%%%%%%%%%%%%%%%%%%%%%%%%%%%%%%%%%%%%%%

%----------------------------------------------------------------------------------------
%	PACKAGES AND OTHER DOCUMENT CONFIGURATIONS
%----------------------------------------------------------------------------------------

\documentclass[12pt]{article} % Default font size is 12pt, it can be changed here

\usepackage{geometry} % Required to change the page size to A4
\geometry{a4paper} % Set the page size to be A4 as opposed to the default US Letter

\usepackage{graphicx} % Required for including pictures

\usepackage{float} % Allows putting an [H] in \begin{figure} to specify the exact location of the figure
\usepackage{wrapfig} % Allows in-line images such as the example fish picture

\usepackage{lipsum} % Used for inserting dummy 'Lorem ipsum' text into the template

\linespread{1.2} % Line spacing

%\setlength\parindent{0pt} % Uncomment to remove all indentation from paragraphs

\graphicspath{{docs/}} % Specifies the directory where pictures are stored

\begin{document}

%----------------------------------------------------------------------------------------
%	INTRODUCTION
%----------------------------------------------------------------------------------------

\section{Background} % Major section



\begin{figure}[H] % Example image
\center{\includegraphics[width=1\linewidth]{{training.jpg}}
\caption{Schematic of educational pathways}
\label{fig:edupathways}
\end{figure}

%----------------------------------------------
%	EDUCATION
%----------------------------------------------

\subsection{Education of anesthetists in Uganda} % Sub-section

After primary school, students in Uganda continue to a four-year program of lower secondary school, after which they take their first set of exams for the Ugandan Certificate of Education (UCE), also known as the "O-level." From here, they have the option of ending their secondary education and begining practical training through certificate programs.  
\cite{UNFPA2009}
% Table R3 Analysis of curricula for different midwifery programmes. "O level 24 and A level is an added advantage"
In nursing, individuals can join certificate programs in nursing, midwifery, or comprehensive nursing (which combines portions of midwifery with traditional nursing).
% new cadre








%----------------------------------------------
%	FACILITIES
%----------------------------------------------

\subsection{Facilities} % Sub-section

Uganda is divided into administrative units as follows (descending population): Districts, Counties, Consitituencies, Sub-Counties, Parishes and Villages. 
% \cite{UBS2017}
% "Uganda is stratified into administrative units namely; Districts, Counties, Consitituencies, Sub-Counties, Parishes and Villages to facilitate service delivery nearer to the people." Page 1 
There are 122 districts, 255 Counties, 296 Consitituencies, 1460 Sub-Counties, and 7467 Parishes. 
% \cite{UBS2017}
% "In total, there are 122 districts as at 1 st July 2017, including Kampala Capital City. Below is a table indicating the number of administrative units at lower levels by region." Page 1

Health facilities are divided into hospitals and health centers. Health centers are divided into four designations, Health Center I through Health Center IV. \cite{UBS2017}
% \cite{UBS2017}
% "Health facilities in Uganda include hospitals and health centres (IV, III and II). The number of functional healthcare facilities in 2015/16 was 5,117 down from 5,205 that were registered in 2012/13 as shown in Table 2.5.1 below." Page 43


There are two national hospitals (Mulago and Butabika Psychiatriac Hospital), four regional referral hospitals, and 139 district, or general, hospitals. Of these, 65 are government owned, 63 are owned by non-profits and 27 are private for-profits.
\cite{MOH-hosp-list}
% \cite{MOH-hosp-list}
%The total number of hospitals (public and private) in Uganda is 155. Of these 2 are National Referral Hospitals (Mulago and Butabika) , 14 are Regional Referral Hospitals (RRHs) and 139 are General Hospitals (GHs). In terms of ownership, 65 are government owned , 63 PNFP and 27 are private. Hospitals are major contributors to outputs of essential clinical care and take up a large volume of human and financial resources. In the financial year 2014 /15 almost similar to the year before, hospitals produced 54% of all inpatient admissions, 19% of total outpatients, and 36% of all deliveries. There are 139 GHs in the country providing; preventive, promotive outpatient curative, maternity, inpatient, emergency surgery and blood transfusion and laboratory services.
In total, in 2015/16, there were 2,932 government-run health facilities, 983 NGO-run, and 1202 privately-operated facilities.
% \cite{UBS2017}
% Table 2.5.1,  Source: Ministry of Health, HMIS 2015/16
Health center II’s are simple outpatient treatment centers which should be staffed by an enrolled nurse, working with a midwife, two nursing assistants and a health assistant. By law, each parish should have at least one functioning health center II. 
% \cite{UBS2017}
% "Out-patient clinic treating common diseases and offering antenatal care. It is supposed to be led by an enrolled nurse, working with a midwife, two nursing assistants and a health assistant. According to the Ugandan government's health policy, every parish is supposed to have a Health Centre II." (Page XX)
Health center III’s are managed by a clinical officer and should include an operating theatre for minor surgeries and cesarean section. Health center IV’s should have a physician on staff, separate wards for men, women, and children, as well as a operating theatre for emergency surgery.
% \cite{UBS2017}
% Health centre with facilities which include an operation room and a section for minor surgery. It is headed by a clinical officer, offers the continuous basic preventive and curative care and provides support supervision of the community and the Health Centre II facilities under its jurisdiction. According to the Ugandan government's health policy, every sub-county is supposed to have a Health Centre III. IV: A mini-hospital that provides the kind of services found at Health Centre III, but in addition has separate wards for men, women, and children in which to admit patients. It should have a senior medical officer and another doctor as well as a theatre for carrying out emergency operations. According to the Ugandan government's health policy, every county or parliamentary constituency is supposed to have a Health Centre IV." (Page XX)
District hospitals (also known as "general hospitals") include all the services of Health Center IV's as well as blood transfusions.
% \cite{UBS2017}
% General hospitals: Health facilities that provide preventive, outpatient curative, maternity, inpatient health services, emergency surgery, blood transfusion, laboratory and other general services. They also provide in-service training, consultation and research in support of community-based health care programmes.
Regional referal hospitals include additional specialized services such as radiology, pathology, and surgical subspecialties.
%In addition to the services offered at the general hospital, these hospitals offer specialist services such as psychiatry, ear, nose and throat, radiology, pathology, ophthalmology as well as higher level surgical and medical services, including teaching and research.
Finally, the national referral hospitals offer comprehensive specialist services. 
%In addition to the services offered at the regional referral hospital, they provide comprehensive specialist services and are involved in teaching and health research.
\cite{UBS2017}











%----------------------------------------------------------------------------------------
%	MAJOR SECTION X - TEMPLATE - UNCOMMENT AND FILL IN
%----------------------------------------------------------------------------------------

%\section{Content Section}

%\subsection{Subsection 1} % Sub-section

% Content

%------------------------------------------------

%\subsection{Subsection 2} % Sub-section

% Content

%----------------------------------------------------------------------------------------
%	CONCLUSION
%----------------------------------------------------------------------------------------

\section{Conclusion} % Major section

\lipsum[12-13]

%----------------------------------------------------------------------------------------
%	BIBLIOGRAPHY
%----------------------------------------------------------------------------------------

\begin{thebibliography}{99} % Bibliography - this is intentionally simple in this template

@ARTICLE{Ugandan_Bureau_of_Statistics2017-nc,
  title  = "{STATISTICAL} {ABSTRACT}",
  author = "{Ugandan Bureau of Statistics}",
  year   =  2017
}

\end{thebibliography}

%----------------------------------------------------------------------------------------

\end{document}